% Homework Template for CS 421
\documentclass{article}

\usepackage{amsmath}  % For mathematical symbols and environments
\usepackage{listings} % For code formatting
\usepackage{color}    % For code coloring
\usepackage{fancyhdr} % For header and footer
\usepackage[a4paper, left=0.8in, right=0.8in, top=1in, bottom=1in]{geometry}
\usepackage{xcolor}   % For custom text colors

\pagestyle{fancy}
\fancyhf{} % Clear all header and footer fields
\fancyhead[L]{Cy Dixon, Kelcee Gabbard} % Left side of the header
\fancyfoot[C]{\thepage}

\author{Cy Dixon, Kelcee Gabbard}
\title{CS 421 - Project 2 Analysis}
\date{\today}

% Code listing settings
\lstset{
    language=Python,
    basicstyle=\ttfamily\footnotesize,
    keywordstyle=\color{blue},
    stringstyle=\color{green},
    commentstyle=\color{gray},
    numbers=left,
    numberstyle=\tiny,
    stepnumber=1,
    numbersep=5pt,
    breaklines=true,
    frame=single,
    tabsize=4,
    showstringspaces=false,
    captionpos=b
}

\begin{document}

\maketitle

\thispagestyle{fancy}

%%%%%%%%%%%%%%%%%%%%%%
% Begin Section 1
%%%%%%%%%%%%%%%%%%%%%%
\section*{1) The Solution}
This project implements a phone number word representation tool by mapping words from a dictionary file into hash tables. The hash tables are used to efficiently store and retrieve words based on their numeric phone number equivalents. 

The solution ensures that phone numbers can be broken into their area code, exchange, and number components to find matching word representations.

%%%%%%%%%%%%%%%%%%%%%%
% Begin Section 2
%%%%%%%%%%%%%%%%%%%%%%
\section*{2) List of Data Structures}
\begin{itemize}
    \item \textbf{HashTable:} Used to store words based on their numeric equivalents. Each hash table uses separate chaining to resolve collisions.
    \item \textbf{ListNode:} Represents nodes in the linked lists for separate chaining in the hash table.
    \item \textbf{Keypad Mapping:} A dictionary mapping phone keypad digits to letters (e.g., '2' maps to 'ABC').
\end{itemize}

%%%%%%%%%%%%%%%%%%%%%%
% Begin Section 3
%%%%%%%%%%%%%%%%%%%%%%
\section*{3) Complexity Analysis}
\subsection*{Loading Words into Hash Tables}
\begin{itemize}
    \item \textbf{Word-to-Number Conversion:} \(O(L)\), where \(L\) is the word length.
    \item \textbf{Insertion into Hash Table:}
        \begin{itemize}
            \item Best Case: \(O(1)\) (no collisions).
            \item Worst Case: \(O(n)\), where \(n\) is the number of collisions in a hash slot.
        \end{itemize}
    \item \textbf{Overall Complexity:} \(O(W \cdot (L + n))\), where \(W\) is the total number of words.
\end{itemize}

\subsection*{Searching Phone Numbers}
\begin{itemize}
    \item \textbf{10-Digit Search:} Best case \(O(1)\), worst case \(O(n)\).
    \item \textbf{7-Digit Search:} Best case \(O(1)\), worst case \(O(n)\).
    \item \textbf{3-Digit and 4-Digit Searches:} Best case \(O(1)\) per table, worst case \(O(n)\).
    \item \textbf{Phone Number Parsing:} \(O(\log_{10} N)\), where \(N\) is the phone number length.
    \item \textbf{Overall Complexity:} \(O(\log_{10} N + 4 \cdot n)\), assuming up to 4 hash tables are checked.
\end{itemize}

%%%%%%%%%%%%%%%%%%%%%%
% Begin Section 4
%%%%%%%%%%%%%%%%%%%%%%
\section*{4) Code}

Paste your code below:

\lstinputlisting[language=Python]{hash_phone.py} % Replace 'project2.py' with the file name or comment out and paste code directly below.

% Alternatively, you can paste code directly below this line.
% \begin{lstlisting}
% # Paste your code here
% \end{lstlisting}

\end{document}
